\section*{Modelo Fuzzy da Microrrede}

O modelo dinâmico transladado pode ser descrito como:

\begin{gather*}
  \dot{x}(t) = A(x) \cdot x(t) + B(x) \cdot u(t) + F(x) \cdot \omega(t)
\end{gather*}

com,

\begin{gather}
  % \dot{x}(t) = \begin{bmatrix}
  %   \delta \dot{i}_{L,1} \\[12pt] \delta \dot{i}_{L,2} \\[12pt] \delta \dot{i}_{L_K} \\[12pt]
  %   \delta \dot{v}_{C,1} \\[12pt] \delta \dot{v}_{C,2} \\[12pt] \delta \dot{v}_{C_K}
  % \end{bmatrix} =
  A = \begin{bmatrix}
    \ds - \frac{R_{1,1}}{L_1} & 0                                          & 0                                & \ds - \frac{1}{L_1}                                                  & 0                                                                                             & 0                   \\[12pt]
    0                         & \ds - \frac{R_{2,1}}{L_2}                  & 0                                & 0                                                                    & \ds \left[ \frac{R_{2,1} i_{L,2}^o}{L_2 v_{C,2}^o} - \frac{v_{in,2}^o}{L_2 v_{C,2}^o} \right] & 0                   \\[12pt]
    0                         & 0                                          & - \ds \frac{R_K + R_{EQ}}{L_K}   & \ds \frac{R_{EQ}}{R_{1,2} L_K}                                       & \ds \frac{R_{EQ}}{R_{2,2} L_K}                                                                & - \ds \frac{1}{L_K} \\[12pt]
    \ds \frac{1}{C_1}         & 0                                          & \ds - \frac{R_{EQ}}{R_{1,2} C_1} & \ds - \frac{1}{R_{1,2}C_1} \left[ 1 - \frac{R_{EQ}}{R_{1,2}} \right] & \ds \frac{1}{C_1 \left(R_{1,2} + R_{2,2}\right)}                                              & 0                   \\[12pt]
    0                         & \ds \frac{1}{C_2} \left( 1 - d_2^o \right) & \ds - \frac{R_{EQ}}{R_{1,2} C_2} & \ds \frac{1}{C_2 \left(R_{1,2} + R_{2,2}\right)}                     & \ds - \frac{1}{R_{2,2}C_1} \left[ 1 - \frac{R_{EQ}}{R_{2,2}} \right]                          & 0                   \\[12pt]
    0                         & 0                                          & \ds \frac{1}{C_K}                & 0                                                                    & 0                                                                                             & a_{66}
  \end{bmatrix}, \quad
  x(t) = \begin{bmatrix}
    \delta i_{L,1}(t) \\[12pt] \delta i_{L,2}(t) \\[12pt] \delta i_{L_K}(t) \\[12pt]
    \delta v_{C,1}(t) \\[12pt] \delta v_{C,2}(t) \\[12pt] \delta v_{C_K}(t)
  \end{bmatrix}, \notag \\[12pt]
  B = \begin{bmatrix}
    \ds \frac{v_{in,1}}{L_1} & 0                                                                & 0                               \\[12pt]
    0                        & \ds \frac{1}{L_2} \left[ v_{C,2}^o + \delta v_{C,2}(t) \right]   & 0
    \\[12pt]
    0                        & 0                                                                & - \ds \frac{R_{EQ}}{L_K}        \\[12pt]
    0                        & 0                                                                & - \ds \frac{R_{EQ}}{R_{1,2}C_1} \\[12pt]
    0                        & - \ds \frac{1}{C_2} \left[ i_{L,2}^o + \delta i_{L,2}(t) \right] & - \ds \frac{R_{EQ}}{R_{2,2}C_2} \\[12pt]
    0                        & 0                                                                & 0
    \\[12pt]
  \end{bmatrix}, \quad
  u(t) = \begin{bmatrix}
    \delta d_1(t) \\[12pt] \delta d_2(t) \\[12pt] \delta i_{B}(t)
  \end{bmatrix}, \notag \\[12pt]
  F = \begin{bmatrix}
     & 0
    \\[12pt]
     & 0
    \\[12pt]
     & 0
    \\[12pt]
     & 0
    \\[12pt]
     & 0
    \\[12pt]
     & - \ds \frac{v_{C_K}^o}{Cv_{C_K}^o \left[ v_{C_K}^o + \delta v_{C_K}(t) \right]}
    \\[12pt]
  \end{bmatrix}, \quad
  \omega(t) = \begin{bmatrix}
    \delta v_{in,1}(t) \\[12pt] \delta v_{in,2}(t) \\[12pt] \delta P_{cpl}(t)
  \end{bmatrix}.
\end{gather}

Em que:

\begin{equation*}
  a_{66} = \ds \frac{P_{cpl}^o}{Cv_{C_K}^o \left[ v_{C_K}^o + \delta v_{C_K}(t) \right]} - \frac{1}{R_{crl} C_K},
\end{equation*}

\begin{equation*}
  f_{11} = \frac{1}{L_1} \left( \frac{R_{1,1}}{v_{in,1}^o}i_{L,1}^o + \frac{1}{v_{in,1}^o}v_{C,1}^o \right).
\end{equation*}


As variáveis premissas:

\begin{gather*}
  z_1(t) = \frac{1}{Cv_{C_K}^o \left[ v_{C_K}^o + \delta v_{C_K}(t) \right]}
\end{gather*}
\begin{gather*}
  z_2(t) = \delta v_{in,1}(t)
\end{gather*}
\begin{gather*}
  z_3(t) = \delta v_{C,2}(t)
\end{gather*}
\begin{gather}
  z_4(t) = \delta i_{L,2}(t)
\end{gather}

\subsection*{Definição das funções de pertinência}

Termo não linear $z_1(t)$:
\begin{gather*}
  \max_{\delta v_{C_K}(t)} z_1(t) = \frac{1}{Cv_{C_K}^o \left[ v_{C_K}^o + \delta v_{C_K}(t) \right]} = q_1 \quad ; \quad \min \delta v_{C_K}(t)
\end{gather*}

\begin{gather}
  \min_{\delta v_{C_K}(t)} z_1(t) = \frac{1}{Cv_{C_K}^o \left[ v_{C_K}^o + \delta v_{C_K}(t) \right]} = q_2 \quad ; \quad \max \delta v_{C_K}(t)
\end{gather}

$z_1(t)$ pode ser definido:

\begin{gather*}
  z_1(t) = \sum\limits_{i=1}^{2} E_i(z_1(t))q_i
\end{gather*}

\begin{gather}\label{eq:z1_definition_by_membership_functions}
  E_1(z_1(t))q_1 + E_2(z_1(t))q_2 = z_1(t)
\end{gather}

O somatório das funções de pertinência devem ser iguais a 1, então:

\begin{gather}\label{eq:z1_membership_functions_sum}
  E_1(z_1(t)) + E_2(z_1(t)) = 1
\end{gather}

Usando as equações~\ref{eq:z1_definition_by_membership_functions} e~\ref{eq:z1_membership_functions_sum}:

\begin{gather*}
  E_1(z_1(t))q_1 = z_1(t) - E_2(z_1(t))q_2
\end{gather*}
\begin{gather*}
  E_1(z_1(t)) = \frac{z_1(t) - E_2(z_1(t))q_2}{q_1}
\end{gather*}
\begin{gather*}
  E_2(z_1(t)) + \frac{z_1(t) - E_2(z_1(t))q_2}{q_1} = 1
\end{gather*}
\begin{gather*}
  E_2(z_1(t))\left(q_1 - q_2 \right) = q_1 - z_1(t)
\end{gather*}
\begin{gather}
  E_2(z_1(t)) = \frac{q_1 - z_1(t)}{q_1 - q_2}
\end{gather}
\begin{gather*}
  E_1(z_1(t)) = 1 - E_2(z_1(t))
\end{gather*}
\begin{gather}
  E_1(z_1(t)) = \frac{z_1(t) - q_2}{q_1 - q_2}
\end{gather}

As funções de pertinência dos demais termos podem ser obtidas de forma análoga.

\vspace{0.5cm}
Termo não linear $z_2(t)$:

\begin{gather*}
  \max_{ v_{in,1}(t)} z_2(t) = \max \delta v_{in,1}(t) = b_1
\end{gather*}

\begin{gather}
  \min_{ v_{in,1}(t)} z_2(t) = \min \delta v_{in,1}(t) = b_2
\end{gather}

\begin{gather*}
  z_2(t) = \sum\limits_{j=1}^{2} M_j(z_2(t))b_i
\end{gather*}
\begin{gather}
  M_1(z_2(t)) = \frac{z_2(t) - b_2}{b_1 - b_2}
\end{gather}
\begin{gather}
  M_2(z_2(t)) = \frac{b_1 - z_2(t)}{b_1 - b_2}
\end{gather}

Termo não linear $z_3(t)$:

\begin{gather*}
  \max_{\delta v_{C,2}(t)} z_3(t) = \max \delta v_{C,2}(t) = c_1
\end{gather*}

\begin{gather}
  \min_{\delta v_{C,2}(t)} z_3(t) = \min \delta v_{C,2}(t) = c_2
\end{gather}

\begin{gather*}
  z_3(t) = \sum\limits_{k=1}^{2} N_k(z_3(t))c_i
\end{gather*}
\begin{gather}
  N_1(z_3(t)) = \frac{z_3(t) - c_2}{c_1 - c_2}
\end{gather}
\begin{gather}
  N_2(z_3(t)) = \frac{c_1 - z_3(t)}{c_1 - c_2}
\end{gather}

Termo não linear $z_4(t)$:

\begin{gather*}
  \max_{\delta i_{L,2}(t)} z_4(t) = \max \delta i_{L,2}(t) = d_1
\end{gather*}

\begin{gather}
  \min_{\delta i_{L,2}(t)} z_4(t) = \min \delta i_{L,2}(t) = d_2
\end{gather}

\begin{gather*}
  z_4(t) = \sum\limits_{l=1}^{2} S_l(z_4(t))d_i
\end{gather*}
\begin{gather}
  S_1(z_4(t)) = \frac{z_4(t) - d_2}{d_1 - d_2}
\end{gather}
\begin{gather}
  S_2(z_4(t)) = \frac{d_1 - z_4`(t)}{d_1 - d_2}
\end{gather}

O modelo fuzzy da microrrede é representado como:

% \begin{gather}
%     \dot{x}(t) =
%   \sum\limits_{i=1}^{2} \sum\limits_{j=1}^{2} \sum\limits_{k=1}^{2} \sum\limits_{l=1}^{2} E_i(z_1(t))M_j(z_2(t))N_k(z_3(t))S_l(z_4(t))
%   \notag \\[12pt] \times
%   \{
%     \begin{bmatrix}
%     \ds - \frac{R_{1,1}}{L_1} & 0                         & 0                                         & \ds - \frac{1}{L_1}                  & 0                                    & 0                           \\[12pt]
%     0                         & \ds - \frac{R_{2,1}}{L_2} & 0                                         & 0                                    & \ds - \left[ \frac{R_{1,1} i_{L,2}^o}{L_2 v_{C,2}^o} - \frac{v_{in,2}^o}{L_2 v_{C,2}^o} \right]                  & 0                           \\[12pt]
%     0                         & 0                         & - \ds \frac{R_K + R_{EQ}}{L_K}   & \ds \frac{R_{EQ}}{R_{1,2} L_K}              & \ds \frac{R_{EQ}}{R_{2,2} L_K}              & - \ds \frac{1}{L_K}         \\[12pt]
%     \ds \frac{1}{C_1}         & 0                         & \ds - \frac{R_{EQ}}{R_{1,2} C_1}                 & \ds - \frac{1}{R_{1,2}C_1} \left[ 1 - \frac{R_{EQ}}{R_{1,2}} \right] & \ds \frac{1}{C_1 \left(R_{1,2} + R_{2,2}\right)}       & 0                           \\[12pt]
%     0                         & \ds \frac{1}{C_2} \left( 1 - d_2^o \right)         & \ds - \frac{R_{EQ}}{R_{1,2} C_2} & \ds \frac{1}{C_2 \left(R_{1,2} + R_{2,2}\right)}      & \ds - \frac{1}{R_{2,2}C_1} \left[ 1 - \frac{R_{EQ}}{R_{1,2}} \right] & 0                           \\[12pt]
%     0                         & 0                         & \ds \frac{1}{C_K}                         & 0                                    & 0                                    & a_{66}
%   \end{bmatrix}
%   \begin{bmatrix}
%     \delta i_{L,1}(t) \\[12pt] \delta i_{L,2}(t) \\[12pt] \delta i_{L_K}(t) \\[12pt]
%     \delta v_{C,1}(t) \\[12pt] \delta v_{C,2}(t) \\[12pt] \delta v_{C_K}(t)
%   \end{bmatrix} \notag \\[12pt] +
%     \begin{bmatrix}
%     \ds \frac{v_{in,1}^o + b_j}{L_1} & 0                 & 0                                                        \\[12pt]
%     0                      & \ds \frac{1}{L_2} \left[ v_{C,2}^o + c_k \right]     & 0                              
%                                     \\[12pt]
%     0                      & 0                 & - \ds \frac{R_{EQ}}{L_K}                                     \\[12pt]
%     0                      & 0                 & - \ds \frac{R_{EQ}}{R_{1,2}C_1}                    \\[12pt]
%     0                      & - \ds \frac{1}{C_2} \left[ i_{L,2}^o + d_l \right]                 & - \ds \frac{R_{EQ}}{R_{2,2}C_2}                                      \\[12pt]
%     0                      & 0                 & 0 
%                                     \\[12pt]
%   \end{bmatrix}
%   \begin{bmatrix}
%     \delta d_1(t) \\[12pt] \delta d_2(t) \\[12pt] \delta i_{B}(t)
%   \end{bmatrix} +
%   \begin{bmatrix}
%       f_{11} & 0                   & 0
%                             \\[12pt]
%       0 & \ds \frac{1}{L_2}   & 0 
%                             \\[12pt]
%       0 & 0                   & 0
%                             \\[12pt]
%       0 & 0                   & 0
%                             \\[12pt]
%       0 & 0                   & 0
%                             \\[12pt]
%       0 & 0                   & - \ds q_i \cdot v_{C_K}^o
%                             \\[12pt]
%   \end{bmatrix}
%   \begin{bmatrix}
%     \delta v_{in,1}(t) \\[12pt] \delta v_{in,2}(t) \\[12pt] \delta P_{cpl}(t)
%   \end{bmatrix}
%   \}
% \end{gather}

% Em que:

% \begin{equation*}
%     a_{66} = \ds q_i \cdot P_{cpl}^o - \frac{1}{R_{crl} C_K},
% \end{equation*} 

% \begin{equation*}
%     f_{11} = \frac{1}{L_1} \left( \frac{R_{1,1}}{v_{in,1}^o}i_{L,1}^o + \frac{1}{v_{in,1}^o}v_{C,1}^o \right).
% \end{equation*}

% ou, 

\begin{gather}
  \dot{x}(t) =
  \sum\limits_{i=1}^{2} \sum\limits_{j=1}^{2} \sum\limits_{k=1}^{2} \sum\limits_{l=1}^{2} E_i(z_1(t))M_j(z_2(t))N_k(z_3(t))S_l(z_4(t))
  \notag \\[12pt] \times
  \left\{
  A_{ijkl}x(t) + B_{ijkl}u(t) + F_{ijkl}\omega(t)
  \right\}
\end{gather}

Os somatórios podem ser reduzidos em um único somatório:

\begin{gather}
  \dot{x}(t) =
  \sum\limits_{\mathbf{i}=1}^{16} h_{\mathbf{i}}(z(t))\left\{ A_{\mathbf{i}}x(t) + B_{\mathbf{i}}u(t) + F_{\mathbf{i}}\omega(t)\right\}
\end{gather}

Em que,

% \begin{gather*}
%     w_1 = \sum\limits_{i=1}^{2}E_i(z_1(t)), \quad w_2 = \sum\limits_{j=1}^{2}M_j(z_2(t)), \quad w_3 = \sum\limits_{k=1}^{2}N_k(z_3(t)), \quad w_4 = \sum\limits_{l=1}^{2}S_l(z_4(t)),
% \end{gather*}

\begin{gather*}
  \mathbf{i}=2^0l+2^1(k-1)+2^2(j-1)+2^3(i-1), \\
  % h_{\mathbf{i}}(z(t)) = \prod_{\mathbf{j}=1}^{4}w_\mathbf{j}
  h_\mathbf{i}(z(t))=E_i\left(z_1(t)\right) M_j\left(z_2(t)\right) N_k\left(z_3(t)\right) S_l\left(z_4(t)\right), \\
  \boldsymbol{A}_\mathbf{i}=\boldsymbol{A}_{i j k l}, \quad \boldsymbol{B}_\mathbf{i}=\boldsymbol{B}_{i j k l}, \quad \boldsymbol{F}_\mathbf{i}=\boldsymbol{B}_{i j k l} .
\end{gather*}

\subsection*{Sistema em malha fechada}
Lei de controle não linear aplicada:

\begin{gather}
  u(t)=K(\hat{x}(t)) \hat{x}(t) + L(\hat{x}(t)) \hat{\omega}(t)=\sum_{\mathbf{j}
    % \in \mathbb{B}^p
  } \mathrm{h}_{\mathbf{j}}(\hat{x}(t)) \left( K_{\mathbf{j}} \hat{x}(t)  + L_{\mathbf{j}} \hat{\omega}(t) \right)
\end{gather}

Em que $\hat{x}(t)$ são os dados disponíveis sobre os estados para o controlador.

\vspace{0.5cm}
Considerando o erros de transmissão:

\begin{gather}
  e_x(t)=\hat{x}(t)-x(t), \quad e_\omega(t)=\hat{\omega}(t)-\omega(t)
\end{gather}

O sistema em malha fechada é

\begin{gather}
  \dot{x}(t) = \sum_{\mathbf{i}} \sum_{\mathbf{j}} h_\mathbf{i}(x(t)) h_\mathbf{j}(\hat{x}) \left\{ (A_{\mathbf{i}} + B_{\mathbf{i}}K_{\mathbf{j}})x + (F_{\mathbf{i}} + B_{\mathbf{i}}L_{\mathbf{j}})\omega + B_{\mathbf{i}}K_{\mathbf{j}}e_x + B_{\mathbf{j}}L_{\mathbf{j}}e_\omega \right\}.
\end{gather}
